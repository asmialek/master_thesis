\subsection{Control Methods}\label{sec:control}
    % \dots\textit{introduction}\dots
    In following sections all control methods inplemented in \ac{scars} are described, along with their implemenatation. Furthermore, the tools available in the toolbox are presented. 
    
    \subsubsection{PID Controoler}
        % \dots\textit{description}\dots
        \ac{pid} is a feedback control loop method, widely used in most industrial applications where the simplicity of the design in of importance. 
        
        

        The controller can be set up to be only proportional, integral or derivative controller, or any combination of these modes. In that case, the gain values for unused modes have to be set to zero.

        In \ac{scars} the input of PID Controller block is error signal and the output is control signal.
        
        \textbf{Attitude vs velocity control}

            \dots\textit{description}\dots
    s
    \subsubsection{LQR}\label{sec:lqr}
        % \dots\textit{description}\dots
        \ac{lqr} is an optimal control method that uses a solution which in simplest form minimizes the quadratic cost function presented in \autoref{eqn:lqrcost} to ganerate static gain matrix $K$.
        \begin{equation}
            cost = \int{x^TQx+u^TRu}
        \end{equation}\label{eqn:lqrcost}
        \ac{lqr} method requires the state ($Q$) and control ($R$) weighting matrices, which respectively correspond to state and input vectors of the system. They describe the control effort that the controller puts on either minimizing the error in each state or magnitude of each input. Both $Q$ and $R$ matrices are diagonal, and most often are chosen arbitrarly and tuned in iterative process to achieve required controller bahavior.

        To use \ac{lqr} method in \ac{scars} Toolbox, the state-space system of the spacecraft model has to be found first. As mentioned before, this is done by following the linearization process described in \autoref{app:linearization}.

        Implemeting LQR Conroller in \ac{scars} toolbox automates the process for the user, asking only to input $Q$ and $R$ matrices as block's mask parameters.
        %https://www.mathworks.com/videos/trim-linearization-and-control-design-for-an-aircraft-68880.html

    \subsubsection{Quaternion Feedback Control}
        \dots\textit{description}\dots
        % Look into the topic
    
    \subsubsection{Analisys Tools}
        \dots\textit{description}\dots