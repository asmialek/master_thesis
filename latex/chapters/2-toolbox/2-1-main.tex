\section{Spacecraft Control Architecture \protect\\ Rapid Simulator (SCARS) Toolbox}\label{sec:toolbox}
    This chapter consists of description of the toolbox designed as a part of this thesis work. After the following introduction, in Sections \ref{toolbox:objectives}, \ref{toolbox:software} and \ref{toolbox:architecture} the objectives of the toolbox and its high level structures are described. After that, one finds theoretical description of satellite mechanics and coordinated systems, with following descriptions of theoretical principles of each major component of the toolbox and their implementation in MATLAB and Simulink software. At the end, methods of visualization of acquired simulations are discussed.  

    To fulfill the main objective of this thesis, that is to provide the community of beginner control engineers with a satellite control system prototyping toolbox, a self-made solution is proposed. This chapter provides the insight into the architecture of \ac*{scars} Toolbox, a software framework created for purposes of this thesis in MATLAB and Simulink. First the main objectives of that solution are stated, then architecture of \ac{scars} is described, to give the initial description of how the toolbox can be used. In following sections the principles of operation of each major part of the toolbox, and how they were implemented, are presented.

    The inputs of \ac{scars} Toolbox - whether used as a parts library and integrated into own project, or as ready-made modular simulation - are parameters of spacecraft hardware, for example such as the size of the satellite, thrusters operational range, and initial mission parameters like time, Keplerian elements or initial body rates. The outputs of the toolbox are performances of each part and simulated behavior of the whole spacecraft, allowing the user to easily test different designs for their satellites.

\subsection{Objectives}\label{toolbox:objectives}
    The toolbox by itself covers the first two objectives of the the thesis. The following listing further specifies what should be expected of the end product and what features users should be able to find in \ac{scars} Toolbox: 
    \begin{itemize}
        \item A model of orbital dynamics of Earth orbiting satellite;
        \item Models of most common satellite actuators and sensors and parametrize them, so that the actual hardware can be reproduced in simulation using values from datasheets;
        \item Modeled sources of environmental forces and torques, including most sources most relevant for small satellites;
        \item Several most basic control methods;
        \item Simulink Custom Library, with all models masked for quick set up;
        \item Methods of conducting preliminary review of feasibility of used hardware components and control methods;
        \item Interfaces allowing the user to connect the toolbox with visualization software.
    \end{itemize}

\subsection{Choice of software}\label{toolbox:software}
    To fulfill the objective of accessibility and ease of modification, MATLAB family of software was chosen as a framework for developed toolbox. MATLAB is one of the most popular scripting language and with the addition of Simulink software it can become a powerful tool with the ability to set up numerical simulations in short time. MATLAB is taught in most technical universities and there is significant number of both courses available online and materials for self-teaching. For one purpose (described in Section \ref{sec:ksp}) a Python script acting as a dataflow bridge was used, as it was the simplest method to solve a problem described in that chapter. Several other software solutions were used for visualization purposes, with the reasoning presented in Section \ref{sec:visualization}.

    Versatility of MATLAB may be attributed to the number of Add-Ons available for it. \ac{scars} Toolbox uses and requires the following modules:
    \begin{itemize}
        \item Aerospace Toolbox
        \item Navigation Toolbox
        \item CubeSat Simulation Library
        \item Control System Toolbox
        \item Simulink 3D Animation
    \end{itemize}

\subsection{Architecture}\label{toolbox:architecture}
    \ac{scars} is divided into two parts: 1) Parts Library and 2) Modular Simulation. The Parts Library contains Simulink subsystems, which can be connected to form models of various complexity and for multiple scenarios. The latter, a Modular Simulation, can be set up with either MATLAB command line scripts to represent user's spacecraft.

    \subsubsection{Parts Library}
        \ac{scars} Parts Library is a ready to use Simulink Custom Library - a collection of blocks available to use in Simulink models. All blocks in library downloaded alongside \ac{scars} are parametrized, masked and described to ease the integration of library parts into user simulation. The library is divided into specific sections: Satellite Models, Control Algorithms, Actuators, Sensors, Environment, Visualization, Example scenarios and Other blocks.
        % \begin{itemize}
        %     \item Satellite Models
        %     \item Control Algorithms
        %     \item Actuators
        %     \item Sensors
        %     \item Environment
        %     \item Visualization
        %     \item Example scenarios
        %     \item Other blocks 
        % \end{itemize}
% 
        \begin{figure}[H]
            \centering
            \includegraphics[width=1\textwidth]{2-toolbox/scars-library.png}
            \caption{SCARS Parts Library screenshot}
            \label{fig:scars-library}
        \end{figure}

    \subsubsection{Modular Simulation}
        \ac{scars} Modular Simulation is a ready-made Simulink model constructed to provide the user with setup containing complete simulation of the spacecraft. The model can be initialized with prepared script described in \autoref{sec:scripts}. The model is a simulation of cube-shaped satellite, which can be set on specified orbit using various initialization methods, such as Keplerian elements in conjunction with Julian date time or geographical coordinates with velocity and rates in body axes. In the same manner, all actuators and sensors available in \ac{scars} library can be connected to act on the spacecraft.

        \begin{figure}[H]
            \centering
            \includegraphics[width=1\textwidth]{2-toolbox/scars-model.png}
            \caption{SCARS Modular Simulation screenshot}
            \label{fig:scars-model}
        \end{figure}


    \subsubsection{Main Signal Buses}\label{sec:buses}
        To unify the signals transferred to blocks modeled in \ac{scars} Toolbox a pair of Simulink signal buses is proposed - \textbf{SatStates} bus, produced as the output of \textbf{Satellite Dynamics} block and \textbf{Env} bus, collected as an output of \textbf{Environment} block. Since the buses are unified, there is no need for unit or coordinate transformations between, for example, satellite model and sensor model. Following tables present descriptions of signals contained in designed buses. Mentioned reference frames are explained in \autoref{sec:frames}. 

        \begin{table}[H]
            \begin{tabularx}{\textwidth}{llXl}
            \textbf{Name} & \textbf{Unit} & \textbf{Description} & \textbf{Size} \\[0.1cm]\hline\rule{0pt}{1.2\normalbaselineskip}
            V\_ECEF & m/s & Velocity of the body in ECEF frame, in relation to ECEF reference frame & 1x3 \\
            X\_ECEF & m & Position of the body in ECEF reference frame & 1x3 \\
            lla & deg, deg, m & Body latitude, longitude and altitude in reference to Earth's Geographical coordinates & 1x3 \\
            Euler\_NED & rad & Body rotation angles in relation to NED reference frame &  \\
            DCM\_ECI2B & - & Direction Cosine Matrix describing rotation from ECI frame to body frame & 3x3 \\
            DCM\_NED2B & - &  Direction Cosine Matrix describing rotation from NED frame to body frame & 3x3  \\
            DCM\_ECEF2NED & - & Direction Cosine Matrix describing rotation from ECEF frame to NED frame & 3x3  \\
            DCM\_ECEF2B & - &  Direction Cosine Matrix describing rotation from ECEF frame to body frame & 3x3  \\
            V\_B & m/s & Body acceleration in relation to its inertial reference frame &  1x3 \\
            Omega\_NED & rad/s & Body rotation rate in reference to NED reference frame & 1x3  \\
            Omega\_B & rad/s & Body rotation rate in reference to its inertial reference frame  & 1x3 \\
            Euler\_B & rad & Body's rotation angles in relation to its initial position  & 1x3 \\
            dOmega\_B/dt & rad/$s^2$ & Derivative of body rotation rate in reference to its inertial reference frame & 1x3 \\
            A\_B & m/$s^2$ & Body acceleration in relation to its inertial reference frame & 1x3
            \end{tabularx}
            \caption{\textbf{SatStates} bus signals description}
        \end{table}

        Since \textbf{SatStates} bus contains ideal values of satellite states, it is used by environment models and models that include mechanical relations between satellite's frame and their hardware components. For other applications, like creation of control loop, it is advised to use sensor models instead.

        
        \begin{table}[H]
            \begin{tabularx}{\textwidth}{llXl}
            \textbf{Name} & \textbf{Unit} & \textbf{Description} & \textbf{Size} \\[0.1cm] \hline \rule{0pt}{1.2\normalbaselineskip}
            Magnetic Field [nT] & $nT$ & Strength of the magnetic field on Earth's orbit, depending on altitude, position and time & 1x3 \\
            Environment Force [N]& $N$ & Force acting on a spacecraft - a sum of gravitational pull and atmospheric drag & 1x3 \\
            Sun's Position [km] & $km$ & Position of the Sun with reference to Earth, in ECEF frame & 1x3 \\
            Atmosphere Density [km/m3]& $kg/m^3$ & Density of atoms in partial atmosphere at body's altitude & 1
            \end{tabularx}
            \caption{\textbf{Env} bus signals description}
        \end{table}

        \textbf{Env} signal bus contains information about parameters of the environment at spacecraft coordinates. Although the values are derived from used models, they should be treated as ideal values. Therefore they are used for applications where the parameter directly influences the bahaviour of the model, like in magnetorquer model described in \autoref{sec:magnetorquer}. For comparison, in B-Dot controller described in \autoref{sec:bdot} it is suggested to use a sensor model instead and place it between \textbf{Env} block and actuator or \ac{obc} model.  