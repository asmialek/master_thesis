\section*{Abstract}

% The process of effective spacecraft project management, from the conception of the idea, through production, to disposal, features high costs and often various unpredictable risks. Due to this, a project life cycle is usually divided into distinct phases, allowing for introduction of conducting product reviews within rigid time-frames.

Spacecraft project management calls for division of project lifetime into phases, with specific goals to be fulfilled at the end of each phase. During first few phases a \ac*{pdr} has to be conducted, after which top-level hardware design is not to be changed. This thesis describes a process of creating and demonstrates a software framework supporting teams building small satellites - typically CubeSat student projects - during initial phases of conceptual design, mission planning, and selection and sizing of hardware components. The scope of the thesis covers review of available tools for satellite mission and control system design, then it proposes a self-made MATLAB/Simulink toolbox - Spacecraft Control Architecture Rapid Simulator (SCARS) Toolbox, as a open source tool with gentle learning curve and ease of reverse engineering approach. In further parts of the thesis examples of usage are provided, and conclusions and descriptions of problems are presented. In the end, this thesis should not only serve as a description of SCARS toolbox, but also as an insight into the task of building a small satellite simulation.\\[0.3cm]

\newhang{\textbf{Keywords: }}spacecraft, satellite, AOCS, ADCS, control design, MATLAB,\\ Simulink, toolbox, software, prototyping

% Projektový management kosmických lodí vyžaduje rozdělení životnosti projektu do fází, přičemž konkrétní cíle musí být splněny na konci každé fáze. Během několika prvních fází musí být provedeno předběžné přezkoumání návrhu, po kterém se nemění návrh hardwaru nejvyšší úrovně. Tato práce popisuje proces vytváření a demonstrace softwarového rámce podporujícího týmy vytvářející malé satelity - obvykle studentské projekty CubeSat - během počátečních fází koncepčního návrhu, plánování misí a výběru a dimenzování hardwarových komponent. Předmětem diplomové práce je přehled dostupných nástrojů pro navrhování družicových misí a řídicích systémů, dále pak vlastní nástroj MATLAB / Simulink toolbox - Spacecraft Control Architecture Rapid Simulator (SCARS), jako open source nástroj s jemnou křivkou učení a snadnost reverzního inženýrského přístupu. V dalších částech práce jsou uvedeny příklady použití a jsou uvedeny závěry a popisy problémů. Nakonec by tato práce neměla sloužit pouze jako popis SCARS toolboxu, ale také jako vhled do úkolu vytvoření malé satelitní simulace.

% \newhang{\textbf{Klíčová slova: }}kosmická loď, satelit, AOCS, ADCS, návrh řízení, MATLAB, Simulink, sada nástrojů, software, prototypování