\documentclass{article}
\usepackage[utf8]{inputenc}

\usepackage{sectsty} % For pagebreak after each section
    \sectionfont{\clearpage}
    \subsectionfont{\clearpage}
    \subsubsectionfont{\clearpage}

\title{Master Thesis}
\author{Adam Śmiałek}
\date{February 2020}

\begin{document}
% \maketitle    
\makeatletter
\renewcommand\@dotsep{240}   % default value 4.5
\makeatother
\tableofcontents
\clearpage


\section*{Notes}
\begin{itemize}
    \item Aims:
    \begin{itemize}
        \item To provide a toolbox/rapid prototyping software
        \item For use for smaller projects, like CubeSat teams, without access to commercial tools
        \item For use for first Feasibility Study, or testing and comparison of different acutators
        \item For learning purposes (i.e. analogue ) 
    \end{itemize}
    \item Add list of example/know actuators/sensors
    \item Parameters of the block (for example power consumption)
    \item Add methods of analisys of the control systems, ie. bode, pole plots 
    \item Already existing tools:
    \begin{itemize}
            \item MATLAB CubeSat Simulation Library
            \item PrincetonSATELLITE Spacecraft Control Toolbox
            \item PROPAT Toolbox
            \item GAST Toolbox
          \end{itemize}
    \item What would differ SCARS from these tools:
    \begin{itemize}
            \item Ease of use
            \item Limited to only certain cases
            \item Open source
            \item Good documentation
        \end{itemize}
\end{itemize}

\section{Introduction}
\subsection{Aim and scope}
\subsection{Prototyping tools}
\subsection{Small spacecraft simulations}
\subsubsection{Already existing tools}
\section{Spacecraft Control Architecture Rapid Simulator}
\subsection{Architecture}
\subsection{Orbit dynamics}
\subsection{Acutators}
\subsubsection{Thrusters}
\subsubsection{Reaction Wheels}
\subsubsection{Gimbaled Momentum Wheel}
\subsubsection{Gimbaled Momentum Wheel}
\subsection{Sensors?}
\subsubsection{Infra-Red, Optical and Radar Sensors}
\subsubsection{Orbital Gyrocompassing}
\subsubsection{Gyros}
\subsubsection{Inertial Measurement Units}
\subsection{Control Methods}
\subsubsection{PID}
\subsubsection{LQR}
\subsubsection{Analisys Tools}
\subsection{Visualization Tools}
\subsubsection{MATLAB Virtual World}
\subsubsection{FlightGear}
\subsubsection{Orbiter}
\subsubsection{Kerbal Space Program}
\section{}


\end{document}
